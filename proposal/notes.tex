\documentclass[12pt]{article}
\setlength{\parindent}{0pt}
\title{Project proposal}
\usepackage[final]{pdfpages}
\usepackage{amsmath}
\usepackage{graphicx}
\usepackage[utf8]{inputenc} 
\author{Bryan Walther, Didrik Samdanger, Carlos Ankour Navarro}
\usepackage{listings}
\usepackage{xcolor}
\usepackage{amssymb}
\graphicspath{ {./img/} }
\renewcommand\arraystretch{1.2}

\definecolor{codegreen}{rgb}{0,0.6,0}
\definecolor{codegray}{rgb}{0.5,0.5,0.5}
\definecolor{codepurple}{rgb}{0.58,0,0.82}
\definecolor{backcolour}{rgb}{0.95,0.95,0.92}

\usepackage{hyperref}
\hypersetup{
    colorlinks,
    citecolor=black,
    filecolor=black,
    linkcolor=black,
    urlcolor=black
}

\lstdefinestyle{mystyle}{
    backgroundcolor=\color{backcolour},   
    commentstyle=\color{codegreen},
    keywordstyle=\color{magenta},
    numberstyle=\tiny\color{codegray},
    stringstyle=\color{codepurple},
    basicstyle=\ttfamily\footnotesize,
    breakatwhitespace=false,         
    breaklines=true,                 
    captionpos=b,                    
    keepspaces=true,                 
    numbers=left,                    
    numbersep=5pt,                  
    showspaces=false,                
    showstringspaces=false,
    showtabs=false,                  
    tabsize=2
}

\lstset{style=mystyle}
\lstset{extendedchars=false}


\begin{document}
\maketitle
\section*{The project idea}
The idea is to have two cameras, and estimate the relative pose between the two cameras.
The pose will be estimated from epipolar geometry. The steps for this will be as follows:
\begin{enumerate}
    \item Calibrate the cameras (if necessary)
    \item Establish a set of features in both images
    \item Match the set of features between the images
    \item Filter outliers
    \item Estimate the essential matrix
    \item Estimate the pose from the essential matrix
\end{enumerate}

We could then experiment with different feature descriptors such as ORB or SIFT and compare the results with some ground truth.
This ground truth can be acquired by getting frames which are moved apart from each other with a known distance and direction.
Other parameters could be compared as well, such as the matching algorithm used.
\\ \\
Estimating the pose between two different cameras could be useful for many applications. 
In the case of UVGs for example, we could use the estimated pose to control the speed of a follower vehicle such that it keeps a set distance from a lead vehicle.
\\ \\
All that we might need for this are access to a couple of cameras from the lab with known intrinsic parameters.
\end{document}
