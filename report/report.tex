\documentclass[12pt]{article}
\setlength{\parindent}{0pt}
\title{TEK5030 Project Report}
\usepackage[final]{pdfpages}
\usepackage{amsmath}
\usepackage{graphicx}
\author{Bryan Walther, Didrik Samdanger, Carlos Ankour Navarro}
\usepackage[utf8]{inputenc} 
\usepackage{listings}
\usepackage{xcolor}
\usepackage{amssymb}
\graphicspath{ {./img/} }
\renewcommand\arraystretch{1.2}

\definecolor{codegreen}{rgb}{0,0.6,0}
\definecolor{codegray}{rgb}{0.5,0.5,0.5}
\definecolor{codepurple}{rgb}{0.58,0,0.82}
\definecolor{backcolour}{rgb}{0.95,0.95,0.92}

\usepackage{hyperref}
\hypersetup{
    colorlinks,
    citecolor=black,
    filecolor=black,
    linkcolor=black,
    urlcolor=black
}

\lstdefinestyle{mystyle}{
    backgroundcolor=\color{backcolour},   
    commentstyle=\color{codegreen},
    keywordstyle=\color{magenta},
    numberstyle=\tiny\color{codegray},
    stringstyle=\color{codepurple},
    basicstyle=\ttfamily\footnotesize,
    breakatwhitespace=false,         
    breaklines=true,                 
    captionpos=b,                    
    keepspaces=true,                 
    numbers=left,                    
    numbersep=5pt,                  
    showspaces=false,                
    showstringspaces=false,
    showtabs=false,                  
    tabsize=2
}

\lstset{style=mystyle}
\lstset{extendedchars=false}


\begin{document}
\maketitle

\section{Introduction}
% Explain what monocular depth estimation is, and the current methods of doing this using CNNs.
% Mention MiDaS and ZoeDepth
% Mention how MiDaS only provides relative depth without scale.
% ZoeDepth attempts to predict depth in meters, but results are not great.
% Introduce the idea of correcting the depth map given some known information about the scene, using traffic footage as a test case.
\section{Methods}
% Explain the approach for getting the correct scale of a depth map by detecting license plates with known real world dimensions.
% Give outline of the procedure.
% Include images of the depth map from both MiDaS and ZoeDepth
% Mention how the ZoeDepth maps appear to have better resolution, which is why we chose these maps as initial maps.
% Mention the downside being that its much slower.
% Include images of the vehicle detections and the cropped images
\section{Results}
% Start by showing the parking lot example, where we already calibrate the camera and have the focal length.

% Show the dashcam example, explain we do not calibrate the camera, and only manually tweak the focal length until we get somewhat reasonable results.
% Only used to show a use case for this.
\section{Discussion}
% Explain how given we have a calibrated camera, and the focal length of the camera, the results seem pretty good for low precision use cases.
% Explain some of the issues:
%   - For this to work, we need good and stable license plate detections.
%   - The issue of perspective when viewing the license plates, could be resolved through filtering or perspective using homography.
%   - Cannot run realtime without a GPU
%   - Poorly optimized
\section{Conclusion}
% Just summarize the results and the discussion.
\end{document}

\end{document}
